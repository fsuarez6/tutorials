\documentclass{memoir}
% Margins
\setulmarginsandblock{2.5cm}{2.5cm}{*}
\setlrmarginsandblock{2cm}{2cm}{*}
\checkandfixthelayout
% Language
\usepackage[english]{babel}
% Graphics
\usepackage[pdftex]{graphicx}
\graphicspath{{./figures/}}
\DeclareGraphicsExtensions{.pdf,.jpg,.jpeg,.png}
\usepackage{todonotes}
% Math
\usepackage[cmex10]{amsmath}
\interdisplaylinepenalty=2500{}
\usepackage{steinmetz}
% Floats
\usepackage{booktabs,multirow}
\usepackage[caption=true,font=footnotesize]{subfig}
\usepackage{wrapfig}
% Useful Macros
\newcommand{\sref}[1]{Section~\ref{#1}}
\newcommand{\code}[1]{\texttt{#1}}
\newcommand\mat[1]{\boldsymbol{#1}}
\newcommand\vect[1]{\boldsymbol{#1}}
\newcommand\matop[2]{\boldsymbol{#1}\left({#2}\right)}
% Tikz
\usepackage{tikz}
\usepackage[american,siunitx]{circuitikz}
\ctikzset{voltage/distance from node=0.8}
\newcommand*\circled[1]{\tikz[baseline=(char.base)]{
            \node[shape=circle,draw,inner sep=2pt] (char) {#1};}}
\tikzset{
    partial ellipse/.style args={#1:#2:#3}{
        insert path={+ (#1:#3) arc (#1:#2:#3)}
    }
}
\newlength\figwidth
% Exercises environment
\usepackage{exsheets}
\SetupExSheets{
  question/print = true ,
  solution/print = true ,
  counter-format = ch.qu ,
  counter-within = chapter
}
% Acronyms
\usepackage[acronym,shortcuts]{glossaries}
\glsdisablehyper
\newacronym{kcl}{KCL}{Kirchhoff's Current Law}
\newacronym{kvl}{KVL}{Kirchhoff's Voltage Law}
% Memoir customization
\makeatletter
\renewcommand\memendofchapterhook{%
  \clearpage\m@mindentafterchapter\@afterheading}
\makeatother
% Reviewing
\usepackage{color}
\usepackage{showframe}
\renewcommand{\ShowFrameColor}{\color{yellow}}
\renewcommand{\ShowFrameLinethickness}{0.1pt}


\title{MA2009: Introduction to Electrical Circuit and Electronic Device}
\author{Francisco Su\'{a}rez-Ruiz}

\begin{document}

\maketitle

\chapter{Circuits Fundamentals}

\begin{question}
  \textbf{Kirchhoff's Laws}
  \begin{figure}[h]
    \centering
    \begin{circuitikz}[scale=0.75,transform shape] \draw
      (0,0) node[ground] {} to[short,*-]  (0,1)
            to[V,v=5<\volt>]    (0,3) -- (0,4) -- (1,4)
            to[R,v=3<\volt>]    (3,4) -- (4,4) to[short,*-] (4,3)
            to[R,v=$v_2$]       (4,1) -- (4,0) -- (0,0)
      (5,4) to[R=$R$,v>=10<\volt>]    (7,4)
      (8,1) to[I=3<\ampere>,v>=$v_1$] (8,3) -- (8,4) -- (7,4)
      (4,4) -- (5,4)
      (8,1) -- (8,0) to[short,-*] (4,0);
    \end{circuitikz}
    \caption{}
    \label{fig:T1-1}
  \end{figure}
  \begin{enumerate}
    \item Identify loop and meshes in the circuit of \fref{fig:T1-1}.
    \item Apply KVL to find voltage $v_1$ and $v_2$.
  \end{enumerate}
\end{question}

\begin{solution}
 The answer goes here...
\end{solution}

\end{document}
